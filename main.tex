\documentclass{scrartcl}

\usepackage[utf8]{inputenc}
\usepackage{hyperref}
\usepackage{url}
\usepackage{graphicx}
\usepackage{amssymb}
\usepackage[T1]{fontenc}
\usepackage{blkarray}
\usepackage{caption}
\usepackage{subcaption}
\usepackage{biblatex}
\addbibresource{bibliography.bib} 


\newcommand{\emailaddr}[1]{\href{mailto:#1}{\texttt{#1}}}


\title{EDAM
\\
Ontology of bioscientific data analysis and data management
\\
\begin{small} 
  Semantic Web - 
  MSc Computer Engineering and Science
\end{small}
}
\author{
    \emailaddr{davide.domini@studio.unibo.it}
}
\date{March 2023}

\begin{document}

\maketitle

\begin{abstract}
  The EDAM ontology is a comprehensive and structured resource for the classification and organization of bioinformatics data, 
    tools, and concepts. It encompasses over 2,000 terms across multiple domains, including data types, data formats, software tools, 
    and data analysis methods. The ontology is designed to facilitate the integration and sharing of bioinformatics resources and knowledge 
    across various disciplines, enabling researchers to more effectively collaborate and build upon one another's work.

  EDAM's structure is built around a set of hierarchical relationships between terms, allowing users to navigate between related concepts 
    and find relevant resources more easily. Additionally, EDAM provides a rich set of metadata for each term, including definitions, synonyms,
    and cross-references to related terms and resources.
  
  The ontology is regularly updated by a community of experts and is freely available for use and integration with other bioinformatics 
    resources. 
\end{abstract}


\newpage
\tableofcontents

\newpage
\listoffigures

\newpage

\section{Overview}

\section{EDAM Core}

\section{Usecases}


%Bibliography
\newpage
\addcontentsline{toc}{section}{Bibliography}
\printbibliography %Prints bibliography


\end{document}